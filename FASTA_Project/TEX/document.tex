\documentclass[a4paper]{article}
\usepackage[14pt]{extsizes} % для того чтобы задать нестандартный 14-ый размер шрифта
\usepackage[utf8]{inputenc}
\usepackage[russian]{babel}
\usepackage{setspace,amsmath}
\usepackage[left=20mm, top=15mm, right=15mm, bottom=15mm, nohead, footskip=10mm]{geometry} % настройки полей документа

\begin{document} % начало документа
	
	% НАЧАЛО ТИТУЛЬНОГО ЛИСТА
	\begin{center}
		\hfill \break
		\hfill \break
		\footnotesize{ФЕДЕРАЛЬНОЕ ГОСУДАРСТВЕННОЕ БЮДЖЕТНОЕ ОБРАЗОВАТЕЛЬНОЕ УЧРЕЖДЕНИЕ}\\ 
		\footnotesize{ВЫСШЕГО ПРОФЕССИОНАЛЬНОГО ОБРАЗОВАНИЯ}\\
		\small{\textbf{«Санкт-Петербургский Государственный Университет»}}\\
		\hfill \break
		\hfill \break
		\normalsize{Математико-механический факультет}\\
		\hfill \break
		\hfill \break
		\normalsize{Кафедра веселых приключений}\\
		\hfill\break
		\hfill \break
		\hfill \break
		\hfill \break
		\hfill \break
		\large{Техническое задание на разработку компьютерной программы для работы с файлами типа FASTA}\\
		\hfill \break
		\hfill \break
		\hfill \break
		\hfill \break
		\hfill \break
		\hfill \break
	\end{center}
	
	\hfill \break
	
	\normalsize{ 
		\begin{tabular}{cccc}
			
			Выполнил & \underline{\hspace{3cm}} & &Т.М. Рахимов \\\\
			Принял & \underline{\hspace{3cm}}& &  В.И. Гориховский \\\\
		\end{tabular}
	}\\
	\hfill \break
	\hfill \break
	\hfill \break
	\hfill \break
	\hfill \break
	\hfill \break
	\begin{center} Санкт-Петербург 2017 \end{center}
	\thispagestyle{empty} % выключаем отображение номера для этой страницы
	
	% КОНЕЦ ТИТУЛЬНОГО ЛИСТА
	
	\newpage
	
	\tableofcontents % Вывод содержания
	\newpage
	\begin{abstract}
	\end{abstract}
	
	\section{Назаначение программы}
	
	\section{Терминология}
	
	\section{Почему и как оно работает или не работает}
	\subsection{Почему?}
	
	\subsection{Как?}
	
	
	
\end{document}  % КОНЕЦ ДОКУМЕНТА !
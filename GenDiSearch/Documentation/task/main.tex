\documentclass{article}
\usepackage{cmap} % Улучшенный поиск русских слов в полученном pdf-файле
\usepackage[T2A]{fontenc} % Поддержка русских букв
\usepackage[utf8]{inputenc} % Кодировка utf8
\usepackage[english, russian]{babel} % Языки: русский, английский
\usepackage{xcolor}
\usepackage{hyperref}

\renewcommand{\labelenumii}{\arabic{enumi}.\arabic{enumii}.}

\begin{document}
    \begin{titlepage}
        \newpage
            \begin{center}
                \Large Библиотека для поиска генетических заболеваний:\\
                Техническое задание
            \end{center}
    \end{titlepage}
    
    \part{Техническое задание}
        \begin{enumerate}
          \item \textbf{Наименование}: \\Библиотека для поиска генетических заболеваний
          \item \textbf{Назначение}: \\Обработка больших массивов файлов формата FASTA и посик мутаций в генах, характерных для тех или иных заболеваний         
          \item \textbf{Функции}:
          \begin{enumerate}
              \item \emph{Чтение из файла}: \\Чтение файлов из директории для дальнейшей обработки
              \item \emph{Запись в файл}: \\Создание файла с названием файла, id последовательности и информации о наличии болезней
              \item \emph{Скачивание данных}: \\Скачивание фалов с последовательностями, содержащими гены с мутациями, с сайта \href{https://www.ncbi.nlm.nih.gov/nuccore}{NCBI}
              \item \emph{Поиск болезней}: \\Поиск заболеваний, связанных с изменением в структуре генов
          \end{enumerate}
          \item \textbf{Структура}:
         	 \begin{enumerate}
        		\item \emph{Пакет для первичной обработки файлов}
        		\item\emph{Пакет для поиска болезней}
         	 \end{enumerate}
          \item \textbf{Интерфейс}: \\Доступ к основным функциям
        \end{enumerate}

\end{document} 
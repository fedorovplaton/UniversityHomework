\documentclass{article}

\usepackage{cmap} % Улучшенный поиск русских слов в полученном pdf-файле
\usepackage[T2A]{fontenc} % Поддержка русских букв
\usepackage[utf8]{inputenc} % Кодировка utf8
\usepackage[english, russian]{babel} % Языки: русский, английский
\usepackage[left = 2cm, right = 2cm, top = 2cm, bottom = 2cm, bindingoffset = 0cm]{geometry} % Отступы
\usepackage{hyperref}

\begin{document}
    \begin{titlepage}
        \newpage
            \begin{center}
                \Large Библиотека для поиска генетических заболеваний:\\
                Итоги работы
            \end{center}
    \end{titlepage}
    
    \part{Итоги работы}
        \section{Реализованные функции}
            \subsection{\emph{Чтение из файла:}}
            	\begin{itemize}
            		\item Рекурсивное чтение всех FASTA файлов из директории
            	\end{itemize}
             \subsection{\emph{Запись в файл:}}
             	\begin{itemize}
             		\item Создание файла с информайией о последовательности (ID, путь к файлу в котором она хранится, результат поиска болезни) 
             	\end{itemize}
            \subsection{\emph{Скачивание данных:}}
            	\begin{itemize}
            		\item Скачиване одним файлом всех последовательностей, найденных по поисковому запросу на \href{https://www.ncbi.nlm.nih.gov/nuccore}{NCBI}
            	\end{itemize}
             \subsection{\emph{Поиск болезней:}}
             	\begin{itemize}
             		\item Реализован поиск болезни Гентнигтона 
             	\end{itemize}
        \section {Структура:}
        	\subsection{\emph{Пакет fileService:}}
            	\begin{itemize}
            		\item FileReader.java осуществляет чтение данных из файлов и директорий
                    \item FileDownloader.java скачивает данные с сайта \href{https://www.ncbi.nlm.nih.gov/nuccore}{NCBI}
                    \item FileSaver.java сохраняет собранную статистику в файл
            	\end{itemize}
             \subsection{\emph{Пакет diseases:}}
             	\begin{itemize}
             		\item Huntington.java поиск изменений гена, характерных для болезни Гентингтона 
             	\end{itemize}
       	\section{\emph{Перспективы:}}
        	\begin{itemize}
        		\item Добавление классов для поиска других заболеваний
                \item Добавление удобного пользавотельского интерфейса
                \item Обработка последовательностей, хранящихся в других базах данных
        	\end{itemize}
            
\end{document} 